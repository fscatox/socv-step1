\section{Conclusion} \label{sec:conclusion}
In this analysis, I explored the importance of early bug detection in the design cycle and the crucial role verification engineers play in ensuring compliance with design specifications. The shortcomings of the directed testing approach are overcome with more advanced methodologies, built upon the features of the \sv language. Constrained-random stimuli, functional coverage and code reuse through high-level abstractions are essential tools for managing complexity and achieving results throughout the verification plan.

Standardized verification methodologies target an experienced user base, already proficient in the way various programming paradigms can help create a highly reusable and flexible testbench framework and aware of the many limitations of directed tests. This work can be seen as an introductory step, providing a bridge between traditional directed testing approaches and the adoption of more advanced methodologies, by leveraging simple object-oriented and generic programming principles. Furthermore, the discussed testbench architecture is general enough to be easily adaptable to typical designs of digital systems courses.

As an example of effectiveness, the hands-on experience with the \ac{alu} unearthed a bug that went overlooked in simpler directed tests, which was partly tied to the representation width of native \vhdl integer types.